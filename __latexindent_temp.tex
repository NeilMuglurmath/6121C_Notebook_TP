\documentclass{article}

	\usepackage{amsmath} %for equatons
	\usepackage{graphicx} %for images
	\usepackage{fancyhdr} %for header

	\pagestyle{fancy} %add header to pages
	\lhead{Conestoga High School Robotics} %this will be in header

	\author{Team 6121C}
	\title{The Pioneers Engineering Notebook}
	\date{} % leaves date section empty on title

	%start
	\begin{document}
		\pagenumbering{gobble} % no show page number on title page
		\maketitle % make default title
		\begin{center}
			\centering VEX Turning Point (2018-19) % added info on title page
		\end{center}
		
		% table of contents page
		\pagenumbering{arabic} % start showing page numbers again
		\newpage
			\tableofcontents

		% introduction head page
		\newpage
			\section{Introductory Information} % new section without number

		% start introduction info
		% preface
		\newpage
			\subsection{Preface}
					\begin{flushleft}
						\qquad This notebook is the story of
						team 6121C throughout the VEX 2018-19 season, Turning Point.
						It is designed to be an organized description of our design,
						build, and programming processes.
					\end{flushleft}	
		
		% use of git
		\newpage
			\subsection{Use of Git Version Control}
					\begin{flushleft}
						\qquad Our use of Git has helped us keep track of our files.
						Everything electronic we do (code, CAD, etc.) is tracked using
						Git version control to keep track of the iterations those files
						have gone through.
					\end{flushleft}
		
		% use of LaTeX
		\newpage
			\subsection{Use of \LaTeX}
						\begin{flushleft}
							\qquad This year, we decided to use \LaTeX\ to make
							our notebook. This helps us have a cleaner notebook
							and use Git version control with it as well. \LaTeX\ is
							a typesetting system designed for technical and
							scientific documentation. We feel that using \LaTeX\ helped us
							be more efficient in documenting our team.
						\end{flushleft}
		
		% about the team
		\newpage
			\subsection{About the Team}
				\subsubsection{History of 6121C}
					\begin{flushleft}
						\qquad Team 6121C The Pioneers planted their seeds during
						the VRC 2016-17 season, Starstruck. Head by veteran roboteer
						Neil Muglurmath, they ranked 6th and earned the Tournament
						Semifinalist award in their first regional compeition
						of the season, qualifying them for the state qualifier. At the Eastern PA State
						Championship, they were semifinalists in their division and won
						the Build Award. \\
						\qquad The team stepped up their game in the VRC 2017-18 season,
						In The Zone. The team was comprised of one member, Neil Muglurmath,
						for a large part of the season. In his first tournament of the season,
						he recieved the Tournament Semifinalist award. After a rebuild, he acquired
						both Tournament Champion and Excellence awards in his second regional compeition
						of the season. After improving upon his second iteration, he obtained the Excellence
						Award once again, along with being a Tournament Semifinalist. By now, 6121C had
						qualified for the CREATE U.S. Open Robotics Championship, which consists of 200 VRC
						teams around the country, but unfortunately there was no more space to compete. After
						a rebuild for the Eastern PA State Championship, the team took home the Tournament
						Champion Award, qualifying them for the VEX Robotics World Championship in
						Louisville, Kentucky. Here, they ended with a record of 5-5 and 44th in the Research Division.
					\end{flushleft}
				
				
	\end{document}