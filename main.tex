\documentclass{article}

	\usepackage[utf8]{inputenc}
	\usepackage{amsmath}  % for equatons
	\usepackage{graphicx} % for images
	\usepackage{fancyhdr} % for header
	\usepackage{float}

	\pagestyle{fancy} %add header to pages
	\lhead{Conestoga High School Robotics} %this will be in header

	\author{The Pioneers}
	\title{Team 6121C Engineering Notebook}
	\date{} % leaves date section empty on title

	\newcommand*\NewPage{\newpage\null}
	% optionally - this creates: Figure 1.1 ... Figure 2.2
	\renewcommand{\thefigure}{\arabic{section}.\arabic{figure}}

	%start
	\begin{document}
		\pagenumbering{gobble} % no show page number on title page
		\maketitle % make default title
		\begin{center}
			\centering VEX Turning Point (2018-19) % added info on title page
		\end{center}

		% table of contents page
		\newpage
			\pagenumbering{roman} % roman numerals for page numbers in table of contents
			\tableofcontents

		% introduction head page
		\newpage
			\pagenumbering{arabic} % start showing arabic page numbers
			\setcounter{page}{1} %start from page 1 for actual notebook
			\section{Introductory Information} % new section without number

		% start introduction info
		% preface
		\newpage
			\subsection{Preface}
					\begin{flushleft}
						\qquad This notebook is the story of
						team 6121C throughout the VEX 2018-19 season, Turning Point.
						It is designed to be an organized description of our design,
						build, and programming processes.
					\end{flushleft}

		% use of git
		\newpage
			\subsection{Use of Git Version Control}
					\begin{flushleft}
						\qquad Our use of Git has helped us keep track of our files.
						Everything electronic we do (code, CAD, etc.) is tracked using
						Git version control to keep track of the iterations those files
						have gone through.
					\end{flushleft}

		% use of LaTeX
		\newpage
			\subsection{Use of \LaTeX}
						\begin{flushleft}
							\qquad This year, we decided to use \LaTeX to make
							our notebook. This helps us have a cleaner notebook
							and use Git version control with it as well. \LaTeX is
							a typesetting system designed for technical and
							scientific documentation. We feel that using \LaTeX helped us
							be more efficient in documenting our team.
						\end{flushleft}

		\NewPage{} % new page in between sections

	% about the team
	\newpage
		\section{About the Team}
			\subsection{History of 6121C}
				\begin{flushleft}
					\qquad Team 6121C The Pioneers planted their seeds during
					the VRC 2016-17 season, Starstruck. Head by veteran roboteer
					Neil Muglurmath, they ranked 6th and earned the Tournament
					Semifinalist award in their first regional compeition
					of the season, qualifying them for the state qualifier. At the Eastern PA State
					Championship, they were semifinalists in their division and won
					the Build Award. \\

					\qquad The team stepped up their game in the VRC 2017-18 season,
					In The Zone. It was comprised of one member, Neil Muglurmath,
					for a large part of the season. In his first tournament of the season,
					he recieved the Tournament Semifinalist award. After a rebuild, he acquired
					both Tournament Champion and Excellence awards in his second regional compeition
					of the season. After improving upon his second iteration, he obtained the Excellence
					Award once again, along with being a Tournament Semifinalist. By now, 6121C had
					qualified for the CREATE U.S. Open Robotics Championship, which consists of 200 VRC
					teams around the country, but unfortunately there was no more space to compete. After
					a rebuild for the Eastern PA State Championship, the team took home the Tournament
					Champion Award, qualifying them for the VEX Robotics World Championship in
					Louisville, Kentucky. Here, they ended with a record of 5-5 and 44th in the
					Research Division.
				\end{flushleft}

		%team biographies
		\newpage
			\subsection{Team Biographies}

				% leadership roles
				\subsubsection{Team Leadership}
					\begin{flushleft}
						\begin{itemize}
							\item Neil Muglurmath % name
							\begin{itemize} % roles
								\item Team Captain
								\item President of CHS Robotics
								\item He is head builder and programmer
								\item Is responsible for the purchases of building materials the 6121 series
								needs to be successful
								\item Recruits new members for CHS Robotics
								\item Gives presentations about CHS Robotics to School District board members
								\item Leads CHS Robotics meetings
								\item Plans competetition schedule of CHS Robotics
								\item Assists newer members of the club
								\item Assists middle school teams
								\item Is responsible for keeping a standard of build and programming quality
								\item Head driver
							\end{itemize}
						\end{itemize}
					\end{flushleft}

			% actual biographies
			\newpage
				\subsubsection{Member Biographies}
					\begin{flushleft}
						\begin{itemize}
							\item Neil Muglurmath
							\begin{itemize}
								\item This marks my 5th year working with VEX Robotics, after one year
								of using VEX parts for Science Olympiad, one year on
								VEXMEN Team 81Y: Cypher, and two years on 6121C.
								My main interests include building, computer science,
								and sports. I am now a junior at Conestoga High School.
							\end{itemize}
						\end{itemize}
					\end{flushleft}

		% team communication
		\newpage
			\subsection{Team communication}
				\subsubsection{Schoology}
					\begin{flushleft}
						\qquad Schoology is a Learning Management System used by the TE School District.
						Because students are already used to Schoology's UI,
						we thought it would be a good idea to use Schoology
						for communication within CHS Robotics. In Schoology,
						both the leaders of the club and the mentor are able to
						post updates about club meetings, compeitions, and more.

						\begin{figure}[h!]
							\includegraphics[width=\linewidth]{{images/Example_Post}} % schoology post
							\caption{Example Schoology post from last season.}
							\label{fig:Example_Post}
						\end{figure}
					\end{flushleft}

	\NewPage % new page in between sections

	%game rules
	\newpage
		\section{Game Rules}
			\subsection{Competition Rules}
				\begin{flushleft}
					\qquad VEX Robotics Competition Turning Point is
					played on a 12’x12’ square field. Two (2) Alliances – one (1) “red” and one (1) “blue” – composed of two (2) Teams each, compete in matches consisting of a fifteen (15) second Autonomous Period, followed by a one minute and forty-five second (1:45) Driver Controlled Period. \\

					\qquad The object of the game is to attain a higher score than the opposing Alliance by High Scoring or Low Scoring Caps, Toggling Flags, and by Alliance Parking or Center Parking Robots on the Platforms. \\

					\qquad There are eight (8) Caps, six (6) Posts, nine (9) Flags, twenty (20) Balls, two (2) Alliance Platforms, and one (1) Center Platform. \\

					\qquad Caps can be Low Scored on the field tiles, or High Scored on Posts, for the Alliance whose color is facing up at the end of the match. Flags can be Toggled to red or blue, and are Scored for the Alliance whose color is Toggled at the end of the match. Low Flags can be Toggled by Robots, but High Flags can only be Toggled by Balls. Turning Point is intended to be a back and forth game, no scored object is safe! \\


					\qquad Alliance Platforms can be used for Alliance Parking by Robots of the same color Alliance as the Platform. The Center Platform can be used by Robots from either Alliance for Center Parking. An additional bonus is awarded to the Alliance that has the most total points at the end of the Autonomous Period. \break
				\end{flushleft}

				% competition field picture
				\begin{figure}[h!]
					\includegraphics[width=\linewidth]{{images/HD_Arena_White}}
					\caption{Competition Field setup.}
					\label{fig:HD_Arena_White}
				\end{figure}

			% skills rules
		\newpage
			\subsection{Robot Skills Challenge Rules}
				\begin{flushleft}
					In this challenge, teams will compete in sixty (60) second long matches in an effort to score as many points as possible. These matches consist of Driving Skills Matches, which will be entirely driver controlled, and Programming Skills Matches, which will be autonomous with limited human interaction. Teams will be ranked based on their combined score in the two types of matches. The playing field will be set up similarly to that of a normal VEX Robotics Competition Turning Point match, with some modifications (see $<$RSC3$>$). \break

					\textbf{$<$RSC3$>$} In a Robot Skills Match, all Flags begin the Match Toggled for the blue Alliance, and all Caps start with the blue side facing upwards. A Team’s score for their Robot Skills Match will be determined by how many points are scored for the red Alliance at the end of the Match, i.e. how many Flags are Toggled to red, how many Caps are placed “red-up”, if the Robot is Alliance Parked on the red Alliance Platform, etc.
				\end{flushleft}

				% skills field picture
				\begin{figure}[h!]
					\includegraphics[width=\linewidth]{{images/Skills_Field}}
					\caption{Competition Field setup.}
					\label{fig:Skills_Field}
				\end{figure}


		% scoring
		\newpage
			\subsection{Scoring}
				\begin{center}
					\begin{tabular}{ l | c}
						Each Low Flag Toggled & 1 point \\ \hline
						Each High Flag Toggled & 2 points \\ \hline
						Each Cap Low Scored & 1 point \\ \hline
						Each Cap High Scored & 2 points \\ \hline
						Robot that is Alliance Parked & 3 points \\ \hline
						Robot that is Center Parked & 6 points \\ \hline
						Autonomous Bonus & 4 points
					\end{tabular}
						\break
						\break
						\break

						Highest Possible Score Match: 45 \break \break
						Highest Possible Combined Skills Score: 70
				\end{center}

	% goals and strategy
	\NewPage{}
	\newpage
		\section{Strategy}
			\subsection{Season Goals for 6121C}
				\begin{flushleft}
					\qquad After a season of many matches left to chance, we want to be a much stronger and reliable team this season. Realiability, driver practice, and quick scoring will be the focal points this season. \break
					\qquad To be more specific:
						\begin{itemize}
							\item Be a well-rounded team
							\begin{itemize}
								\item Be strong in all aspects of the game
								\item Have multiple autonomous routines that can achieve different types of scoring
							\end{itemize}
							\item Easily win most matches
							\item Iron out all problems way before competitions
							\item Rank high in skills
							\item Consistently rank in top 2 of qualifying rankings
						\end{itemize}
				\end{flushleft}

			\subsection{Season Goals for CHS Robotics}
				\begin{flushleft}
					\qquad During previous seasons, 6121A and B were not as highly ranked as 6121C. This year, we want to change this. We want the 6121 series to be one of the dominant VRC organizations in competitions. Multiple strong 6121 teams will be beneficial for alliance selection because since we would have worked with the other teams so much during practice, we have a good understanding of how they work. With this knowledge, we can choose other 6121 teams to be alliance partners and work cohesively for the win. \break

					Plan for A and B teams to succeed:
					\begin{itemize}
						\item Give them ideas
						\item Encourage driving practice
						\item Have 1v1 scrimmmages
						\item Show teams videos of matches online
						\item Encourage good build quality
						\item Encourage asking for help
					\end{itemize}
				\end{flushleft}

			\subsection{Game-Specific Strategy}
				\begin{flushleft}
					\qquad Our robot for this year must have fast and consistent scoring in all aspects of the game. To achieve this, one characteristic our robot must have is a fast drive to out-maneuver all defense, but the drive also must have enough torque to stay on the Center Platform. Also, the robot must shoot balls at flags at a faster rate than our opponents. This way, we have control of the flags. Another aspect our robot must have is that it should be able to quickly score caps on the poles and flip caps already on poles. This will minimize the time we spend on caps while giving us control of the caps on the field. All the design choices we consider for a fast robot in compeition will be helpful in Robot Skills as well. \break

					Specific strategies on different aspects of the game:
					\begin{itemize}
						\item Low Flags
						\begin{itemize}
							\item 3 Low Flags worth 1 point each
							\item 3 points total
							\item Robot can just drive forward to score them
							\item Not the main priority
						\end{itemize}

						\item High Flags
						\begin{itemize}
							\item 6 High Flags worth 2 points each
							\item 12 points total
							\item Robot should go up to high flags and do middle and top flag in quick succession, this could help in fast scoring
							\item Main priority
						\end{itemize}

						\item Low Scored Caps
						\begin{itemize}
							\item 8 caps worth 1 point each
							\item 8 points total
							\item Caps should be high scored
							\item never settle for low scored caps
						\end{itemize}

						\item High Scored Caps
						\begin{itemize}
							\item 8 caps worth 2 points each
							\item 16 points total
							\item Main priority along with flags
							\item Worth most of the points
							\item Robot should easily flip caps on poles without having to take them down
						\end{itemize}

						\item Center Parking
						\begin{itemize}
							\item Worth 6 points
							\item We should park as soon as other team goes to park
							\item Should consider that parking needs lots of torque to stay up
							\item Should have no trouble getting up the pipes
						\end{itemize}

						\item Autonomous
						\begin{itemize}
							\item Worth 4 points
							\item We should have a high-scoring but most of all consistent autonomous routine
						\end{itemize}
					\end{itemize}
				\end{flushleft}

		\newpage
			\subsection{Robot Design Strategy}

				\subsubsection{Control System}
					\begin{itemize}

						% cortex
						\item Cortex
						\begin{itemize}
							\item The VEX Cortex control system has been tested by many teams for years. The common problems and solutions are online and it is easy to find support for the Cortex system through the VEX Forum. It is well-known by both CHS Robotics and other teams.

							\item Even though the Cortex is what we are used to using, it has many negatives. The 393 motors are very weak and do not last very long. Also, 6121C's programmer uses a Mac, so programming the Cortex through Mac is very buggy. It could lead to random issues that fix themselves through a re-download of the binary, leading to confusion whether an issue is from the actual source code or not.
						\end{itemize}

						% V5
						\item V5
						\begin{itemize}
							\item This is the first year that the V5 Control System is legal to use in VRC. Although CHS Robotics has no experience with it, the system seems to have many improvements over the Cortex system. It is possible to get more control over the motors and the Brain has many nice features such as ability to play practice matches without using a Competition Control Switch. Also, the motors can take much more than 393 motors. The addition of the Vision Sensor could open up many possibilities for more consistent autonomous routines.

							\item Even though the V5 has these positives, it is still true that V5 has not been used in compeition yet. There could be many unadressed issues in V5. Also, being that V5 is a newly released system, there is not nearly as much information about it online than there is about the Cortex system. Also, since many teams want the V5 system, there could potentially be a backorder on V5 products. Another reason that teams might stray away from V5 is because of the 8-motor limit, as opposed to the 12-motor limit when using the Cortex system.
						\end{itemize}

						\paragraph{Verdict: V5}
						Even though CHS Robotics is used to using the tried-and-true Cortex control system, its negatives outweigh the positives. The Cortex is buggy and unreliable sometimes, and this cannot be an issue for us this year. We chose to use the V5 Control System for this season.
					\end{itemize}

			\newpage
				\subsubsection{Drivetrain}
					\begin{itemize}

						%H-Drive
						\item H-Drive
						\begin{figure}[H]
							\includegraphics[width=\linewidth]{{images/H_Drive}}
							\caption{CAD model of a widely used H-Drive drivetrain design.}
							\label{fig:H_Drive}
						\end{figure}
						\begin{itemize}
							\item The H-Drive drivetrain is the simplest holonomic drive. It allows both vertical and horizontal movement with the easiest build. This drivetrain consists of three modules: two on the left and right for forward and backward movement, and one at the center of gravity on the robot and perpendicular to the two sides for lateral movement. As far as programming goes, the H-Drive
							is the easiest for autonomous movements. It can move in every direction and drives reasonably straight.

							\item The H-Drive also has some negatives. It takes 6 motors; way more than other drivetrains. Also, the center module block an important area for other robot components, such as an intake. Another reason why this drive might not be the best choice is that it can have trouble getting up the platforms.
						\end{itemize}
					\newpage
						\item X-Drive
						\begin{figure}[H]
							\includegraphics[width=\linewidth]{{images/X_Drive}}
							\caption{CAD model of a widely used X-Drive drivetrain design.}
							\label{fig:X_Drive}
						\end{figure}
						\begin{itemize}
							\item The X-Drive is another widely used holonomic drivetrain. It consists of four modules connected 45º to the main chassis frame. As a result of the vector addition, this drivetrain can move in every direction.

							\begin{figure}[H]
								\includegraphics[width=\linewidth]{{images/X_Drive_Vector}}
								\caption{CAD model of a widely used X-Drive drivetrain design.}
								\label{fig:X_Drive_Vector}
							\end{figure}
							\qquad As depicted in Figure \ref{fig:X_Drive_Vector}, the drive base moves forward because of the sum of the individual vectors created by the 4 modules. The same concept applies when the base is going backward or strafing. Unfortunately, this method of driving by vector makes the robot prone to not driving perfectly straight, because if the vectors do not add up properly, the robot will not drive straight.
							\item The module configuration creates a $\sqrt{2}:1$ ratio for speed. As this causes the drivetrain to be faster than other styles of drivetrain with the same ratio, the drivetrain will also have less torque.

							\begin{figure}[H]
								\includegraphics[width=\linewidth]{{images/X_Tank_Turn}}
								\caption{Turning circle of 3 different drive styles.}
								\label{fig:X_Tank_Turn}
							\end{figure}
							\item Another unique aspect of the X-Drive style is its excellent ability to execute point turns. As represented in \ref{fig:X_Tank_Turn}, the wheels on the X-Drive are tangent to the circle the four wheels form, while the wheels are not tangent to the circle on the four-wheel tank drive. The wheels not being tangent to the circle result in lots of motion of the rollers on the omni-directional wheels, which increases friction.

							\item While the X-Drive seems like a solid option, it has also some negatives. Since the wheels are not straight and are instead in a 45º configuration, the robot will have a hard time going up the platforms. Also, since the 45º configuration results in a loss of torque, the robot could have issues staying at the top of the platform. Furthermore, this drive style requires 4 motors, which might not be ideal when we are limited to 8 motors.
						\end{itemize}

						% tank drive
						\item Tank Drive
						\begin{itemize}
							\item In VRC, the Tank Drive drivesystem is the most widely used drivetrain. It is also one of the most simple and effective drivetrains. Because of how simple it is, it can be build in different ways for different needs. Progamming is also simple for this chassis style.
							\begin{figure}
								\includegraphics[width=\linewidth]{{images/Tank_Drive}}
								\caption{Turning circle of 3 different drive styles.}
								\label{fig:Tank_Drive}
							\end{figure}
							\qquad As pictured in \ref{fig:Tank_Drive}, the Tank Drive drivetrain consists of two modules on either side of the robot. \ref{fig:Tank_Drive} is a 6-wheel variant. of the Tank Drive drivetrain.

							\item As good as this drivetrain is, it comes with its own problems. One is that the wheels are not tangent to the turning cirle, as portrayed in \ref{fig:X_Tank_Turn}. Also, build quality must be near perfect for the drivetrain to travel straight.
						\end{itemize}

						\paragraph{Verdict: Tank Drive}
						After thinking through the positives and negatives of the possible drivetrains, we chose to use the Tank Drive. We felt that the simplicity and the effectiveness of this drivetrain outweighed the positives, including strafing, that the other drivetrain choices offered.
					\end{itemize}

				\subsubsection{Launcher}
				\begin{itemize}
					\item Flywheel
					\begin{itemize}
						\item The flywheel was a popular launcher during the last shooting game, VRC Nothing But Net. The assembly generally consists of one or more wheels on an axle and a polycarbonate hood. They spin at very high speeds to launch balls that pass between the wheels and the hood.
						\begin{figure}[H]
							\includegraphics[width=\linewidth]{{images/Flywheel_Example}}
							\caption{1104M's flywheel shooting balls.}
							\label{fig:Flywheel_Example}
						\end{figure}
						\qquad In \ref{fig:Flywheel_Example}, VRC Team 1104M uses their flywheel assembly to shoot balls during the VRC 2015-16 game, Nothing But Net. Their flywheel is powered by 4 393 motors with 100 RPM internal ratios and a 25:1 external ratio for speed. They have a polycarbonate hood behind their flywheel. When the ball passes through the opening between the hood and the wheels, it gets slightly compressed. This helps the flywheel transfer more energy to the ball. The opening height is one of the specifications that a team needs to tune on a flywheel, along with gear ratio, number of wheels, spacing, and more.
						\qquad The flywheel has the capability of shooting balls at many different velocities, allowing the robot to shoot any flag from anywhere on the field. Working alongside the vision sensor, which can align the robot and calculate the required velocity for the flywheel to launch the ball, the robot could consistently hit the flags as soon as it picks up the balls. Another benefit of the flywheel is that it could shoot two balls, in succession, at the high and middle flags. This way, we can quickly control the flags on the field.

						\item The flywheel design is not without its negatives. One major flaw with this launcher type is the time it takes to get back up to speed after launching a ball. After launching, some of the kinetic energy of the flywheel is lost, and it takes time for the flywheel to gain this energy again. Also, in the beginning of a match, the flywheel needs to start spinning from a standstill, and that takes time as well.
					\end{itemize}
				\end{itemize}
	\end{document}