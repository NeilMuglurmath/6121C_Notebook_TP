\documentclass{article}

	\usepackage{amsmath}  % for equatons
	\usepackage{graphicx} % for images
	\usepackage{fancyhdr} % for header

	\pagestyle{fancy} %add header to pages
	\lhead{Conestoga High School Robotics} %this will be in header

	\author{Team 6121C}
	\title{The Pioneers Engineering Notebook}
	\date{} % leaves date section empty on title

	\newcommand*\NewPage{\newpage\null}
	% optionally - this creates: Figure 1.1 ... Figure 2.2
	\renewcommand{\thefigure}{\arabic{section}.\arabic{figure}}

	%start
	\begin{document}
		\pagenumbering{gobble} % no show page number on title page
		\maketitle % make default title
		\begin{center}
			\centering VEX Turning Point (2018-19) % added info on title page
		\end{center}

		% table of contents page
		\newpage
			\pagenumbering{roman} % roman numerals for page numbers in table of contents
			\tableofcontents

		% introduction head page
		\newpage
			\pagenumbering{arabic} % start showing arabic page numbers
			\setcounter{page}{1} %start from page 1 for actual notebook
			\section{Introductory Information} % new section without number

		% start introduction info
		% preface
		\newpage
			\subsection{Preface}
					\begin{flushleft}
						\qquad This notebook is the story of
						team 6121C throughout the VEX 2018-19 season, Turning Point.
						It is designed to be an organized description of our design,
						build, and programming processes.
					\end{flushleft}

		% use of git
		\newpage
			\subsection{Use of Git Version Control}
					\begin{flushleft}
						\qquad Our use of Git has helped us keep track of our files.
						Everything electronic we do (code, CAD, etc.) is tracked using
						Git version control to keep track of the iterations those files
						have gone through.
					\end{flushleft}

		% use of LaTeX
		\newpage
			\subsection{Use of \LaTeX}
						\begin{flushleft}
							\qquad This year, we decided to use \LaTeX\ to make
							our notebook. This helps us have a cleaner notebook
							and use Git version control with it as well. \LaTeX\ is
							a typesetting system designed for technical and
							scientific documentation. We feel that using \LaTeX\ helped us
							be more efficient in documenting our team.
						\end{flushleft}

		\NewPage{} % new page in between sections

	% about the team
	\newpage
		\section{About the Team}
			\subsection{History of 6121C}
				\begin{flushleft}
					\qquad Team 6121C The Pioneers planted their seeds during
					the VRC 2016-17 season, Starstruck. Head by veteran roboteer
					Neil Muglurmath, they ranked 6th and earned the Tournament
					Semifinalist award in their first regional compeition
					of the season, qualifying them for the state qualifier. At the Eastern PA State
					Championship, they were semifinalists in their division and won
					the Build Award. \\

					\qquad The team stepped up their game in the VRC 2017-18 season,
					In The Zone. It was comprised of one member, Neil Muglurmath,
					for a large part of the season. In his first tournament of the season,
					he recieved the Tournament Semifinalist award. After a rebuild, he acquired
					both Tournament Champion and Excellence awards in his second regional compeition
					of the season. After improving upon his second iteration, he obtained the Excellence
					Award once again, along with being a Tournament Semifinalist. By now, 6121C had
					qualified for the CREATE U.S. Open Robotics Championship, which consists of 200 VRC
					teams around the country, but unfortunately there was no more space to compete. After
					a rebuild for the Eastern PA State Championship, the team took home the Tournament
					Champion Award, qualifying them for the VEX Robotics World Championship in
					Louisville, Kentucky. Here, they ended with a record of 5-5 and 44th in the
					Research Division.
				\end{flushleft}

		%team biographies
		\newpage
			\subsection{Team Biographies}

				% leadership roles
				\subsubsection{Team Leadership}
					\begin{flushleft}
						\begin{itemize}
							\item Neil Muglurmath % name
							\begin{itemize} % roles
								\item Team Captain
								\item President of CHS Robotics
								\item He is head builder and programmer
								\item Is responsible for the purchases of building materials the 6121 series
								needs to be successful
								\item Recruits new members for CHS Robotics
								\item Gives presentations about CHS Robotics to School District board members
								\item Leads CHS Robotics meetings
								\item Plans competetition schedule of CHS Robotics
								\item Assists newer members of the club
								\item Assists middle school teams
								\item Is responsible for keeping a standard of build and programming quality
								\item Head driver
							\end{itemize}
						\end{itemize}
					\end{flushleft}

			% actual biographies
			\newpage
				\subsubsection{Member Biographies}
					\begin{flushleft}
						\begin{itemize}
							\item Neil Muglurmath
							\begin{itemize}
								\item This marks my 5th year working with VEX Robotics, after one year
								of using VEX parts for Science Olympiad, one year on
								VEXMEN Team 81Y: Cypher, and two years on 6121C.
								My main interests include building, computer science,
								and sports. I am now a junior at Conestoga High School.
							\end{itemize}
						\end{itemize}
					\end{flushleft}

		% team communication
		\newpage
			\subsection{Team communication}
				\subsubsection{Schoology}
					\begin{flushleft}
						\qquad Schoology is a Learning Management System used by the TE School District.
						Because students are already used to Schoology's UI,
						we thought it would be a good idea to use Schoology
						for communication within CHS Robotics. In Schoology,
						both the leaders of the club and the mentor are able to
						post updates about club meetings, compeitions, and more.

						\begin{figure}[h!]
							\includegraphics[width=\linewidth]{{images/Example_Post}} % schoology post
							\caption{Example Schoology post from last season.}
							\label{fig:Example_Post}
						\end{figure}
					\end{flushleft}

	\NewPage % new page in between sections

	%game rules
	\newpage
		\section{Game Rules}
			\subsection{Competition Rules}
				\begin{flushleft}
					\qquad VEX Robotics Competition Turning Point is
					played on a 12’x12’ square field. Two (2) Alliances – one (1) “red” and one (1) “blue” – composed of two (2) Teams each, compete in matches consisting of a fifteen (15) second Autonomous Period, followed by a one minute and forty-five second (1:45) Driver Controlled Period. \\

					\qquad The object of the game is to attain a higher score than the opposing Alliance by High Scoring or Low Scoring Caps, Toggling Flags, and by Alliance Parking or Center Parking Robots on the Platforms. \\

					\qquad There are eight (8) Caps, six (6) Posts, nine (9) Flags, twenty (20) Balls, two (2) Alliance Platforms, and one (1) Center Platform. \\

					\qquad Caps can be Low Scored on the field tiles, or High Scored on Posts, for the Alliance whose color is facing up at the end of the match. Flags can be Toggled to red or blue, and are Scored for the Alliance whose color is Toggled at the end of the match. Low Flags can be Toggled by Robots, but High Flags can only be Toggled by Balls. Turning Point is intended to be a back and forth game, no scored object is safe! \\


					\qquad Alliance Platforms can be used for Alliance Parking by Robots of the same color Alliance as the Platform. The Center Platform can be used by Robots from either Alliance for Center Parking. An additional bonus is awarded to the Alliance that has the most total points at the end of the Autonomous Period. \break
				\end{flushleft}

				% competition field picture
				\begin{figure}[h!]
					\includegraphics[width=\linewidth]{{images/HD_Arena_White}} % schoology post
					\caption{Competition Field setup.}
					\label{fig:HD_Arena_White}
				\end{figure}

			% skills rules
		\newpage
			\subsection{Robot Skills Challenge Rules}
				\begin{flushleft}
					In this challenge, teams will compete in sixty (60) second long matches in an effort to score as many points as possible. These matches consist of Driving Skills Matches, which will be entirely driver controlled, and Programming Skills Matches, which will be autonomous with limited human interaction. Teams will be ranked based on their combined score in the two types of matches. The playing field will be set up similarly to that of a normal VEX Robotics Competition Turning Point match, with some modifications (see $<$RSC3$>$). \break

					\textbf{$<$RSC3$>$} In a Robot Skills Match, all Flags begin the Match Toggled for the blue Alliance, and all Caps start with the blue side facing upwards. A Team’s score for their Robot Skills Match will be determined by how many points are scored for the red Alliance at the end of the Match, i.e. how many Flags are Toggled to red, how many Caps are placed “red-up”, if the Robot is Alliance Parked on the red Alliance Platform, etc.
				\end{flushleft}

				% skills field picture
				\begin{figure}[h!]
					\includegraphics[width=\linewidth]{{images/Skills_Field}} % schoology post
					\caption{Competition Field setup.}
					\label{fig:Skills_Field}
				\end{figure}


		% scoring
		\newpage
			\subsection{Scoring}
				\begin{center}
					\begin{tabular}{ l | c}
						Each Low Flag Toggled & 1 point \\ \hline
						Each High Flag Toggled & 2 points \\ \hline
						Each Cap Low Scored & 1 point \\ \hline
						Each Cap High Scored & 2 points \\ \hline
						Robot that is Alliance Parked & 3 points \\ \hline
						Robot that is Center Parked & 6 points \\ \hline
						Autonomous Bonus & 4 points
						\end{tabular}
						\break
						\break
						\break

						Highest Possible Score Match: 45 \break \break
						Highest Possible Combined Skills Score: 70
					\end{center}

	% goals and strategy
	\NewPage{}
	\newpage
		\section{Strategy}
			\subsection{Season Goals for 6121C}
				\begin{flushleft}
					\qquad After a season of many matches left to chance, we want to be a much stronger and reliable team this season. Realiability, driver practice, and quick scoring will be the focal points this season. \break
					\qquad To be more specific:
						\begin{itemize}
							\item Be a well-rounded team
							\begin{itemize}
								\item Be strong in all aspects of the game
								\item Have multiple autonomous routines that can achieve different types of scoring
							\end{itemize}
							\item Easily win most matches
							\item Iron out all problems way before competitions
							\item Rank high in skills
							\item Consistently rank in top 2 of qualifying rankings
						\end{itemize}
				\end{flushleft}

			\subsection{Season Goals for CHS Robotics}
				\begin{flushleft}
					\qquad During previous seasons, 6121A and B were not as highly ranked as 6121C. This year, we want to change this. We want the 6121 series to be one of the dominant VRC organizations in competitions. Multiple strong 6121 teams will be beneficial for alliance selection because since we would have worked with the other teams so much during practice, we have a good understanding of how they work. With this knowledge, we can choose other 6121 teams to be alliance partners and work cohesively for the win. \break

					Plan for A and B teams to succeed:
					\begin{itemize}
						\item Give them ideas
						\item Encourage driving practice
						\item Have 1v1 scrimmmages
						\item Show teams videos of matches online
						\item Encourage good build quality
						\item Encourage asking for help
					\end{itemize}
				\end{flushleft}

			\subsection{Game-Specific Strategy}
				\begin{flushleft}
					\qquad Our robot for this year must have fast and consistent scoring in all aspects of the game. To achieve this, one characteristic our robot must have is a fast drive to out-maneuver all defense, but the drive also must have enough torque to stay on the Center Platform. Also, the robot must shoot balls at flags at a faster rate than our opponents. This way, we have control of the flags. Another aspect our robot must have is that it should be able to quickly score caps on the poles and flip caps already on poles. This will minimize the time we spend on caps while giving us control of the caps on the field. All the design choices we consider for a fast robot in compeition will be helpful in Robot Skills as well. \break

					Specific strategies on different aspects of the game:
					\begin{itemize}
						\item Low Flags
						\begin{itemize}
							\item 3 Low Flags worth 1 point each
							\item 3 points total
							\item Robot can just drive forward to score them
							\item Not the main priority
						\end{itemize}

						\item High Flags
						\begin{itemize}
							\item 6 High Flags worth 2 points each
							\item 12 points total
							\item Robot should go up to high flags and do middle and top flag in quick succession, this could help in fast scoring
							\item Main priority
						\end{itemize}

						\item Low Scored Caps
						\begin{itemize}
							\item 8 caps worth 1 point each
							\item 8 points total
							\item Caps should be high scored
							\item never settle for low scored caps
						\end{itemize}

						\item High Scored Caps
						\begin{itemize}
							\item 8 caps worth 2 points each
							\item 16 points total
							\item Main priority along with flags
							\item Worth most of the points
							\item Robot should easily flip caps on poles without having to take them down
						\end{itemize}

						\item Center Parking
						\begin{itemize}
							\item Worth 6 points
							\item We should park as soon as other team goes to park
							\item Should consider that parking needs lots of torque to stay up
							\item Should have no trouble getting up the pipes
						\end{itemize}

						\item Autonomous
						\begin{itemize}
							\item Worth 4 points
							\item We should have a high-scoring but most of all consistent autonomous routine
						\end{itemize}
					\end{itemize}
				\end{flushleft}




	\end{document}